\documentclass[fleqn,xcolor=dvipsnames]{beamer}
%
% Choose how your presentation looks.
%
% For more themes, color themes and font themes, see:
% http://deic.uab.es/~iblanes/beamer_gallery/index_by_theme.html
%
\mode<presentation>
{
  \usetheme{Madrid}      % or try Darmstadt, Madrid, Warsaw, ...
  \usecolortheme{default} % or try albatross, beaver, crane, ...
  \usefonttheme{default}  % or try serif, structurebold, ...
  \setbeamertemplate{navigation symbols}{}
  \setbeamertemplate{caption}[numbered]
} 

\usepackage{amscd,latexsym,amsthm,amsfonts,amssymb,amsmath,amsxtra}
\usepackage[english]{babel}
\usepackage[utf8x]{inputenc}
\usepackage{amsmath, amssymb, amsthm}
\usepackage{ytableau}
\usepackage{graphicx}
\usepackage{tikz}
\usepackage{tikz-cd}
\usepackage{xcolor}
\usepackage{extarrows}
\usepackage{booktabs}

%%%%%%%%%%%%%%%%%%%%%%%%%%%%%%%%%%%%%%%%%%%%%% Theorem style

\newtheorem{thm}{Theorem}[section]
\newtheorem{cor}[thm]{Corollary}
\newtheorem{prop}[thm]{Proposition}
\newtheorem{lem}[thm]{Lemma}
\newtheorem{assump}[thm]{Assumption}
\newtheorem{conj}[thm]{Conjecture}
\newtheorem{rk}[thm]{Remark}
\newtheorem{question}[thm]{Question}
\newtheorem{defn}[thm]{Definition}
\newtheorem{con}[thm]{Construction}
\newtheorem{examp}[thm]{Example}
\newtheorem{notn}[thm]{Notation}
\newtheorem{exer}[thm]{Exercise}

%%%%%%%%%%%%%%%%%%%%%%%%%%%%% Bold Fonts
\newcommand{\blam}{{\boldsymbol{\lambda}}}
\newcommand{\bnu}{{\boldsymbol{\nu}}}
\newcommand{\br}{{\mathbf{r}}}
\newcommand{\bc}{{\mathbf{c}}}
\newcommand{\BA}{{\mathbb {A}}}
\newcommand{\BB}{{\mathbb {B}}}
\newcommand{\BC}{{\mathbb {C}}}
\newcommand{\BD}{{\mathbb {D}}}
\newcommand{\BE}{{\mathbb {E}}}
\newcommand{\BF}{{\mathbb {F}}}
\newcommand{\BG}{{\mathbb {G}}}
\newcommand{\BH}{{\mathbb {H}}}
\newcommand{\BI}{{\mathbb {I}}}
\newcommand{\BJ}{{\mathbb {J}}}
\newcommand{\BK}{{\mathbb {U}}}
\newcommand{\BL}{{\mathbb {L}}}
\newcommand{\BM}{{\mathbb {M}}}
\newcommand{\BN}{{\mathbb {N}}}
\newcommand{\BO}{{\mathbb {O}}}
\newcommand{\BP}{{\mathbb {P}}}
\newcommand{\BQ}{{\mathbb {Q}}}
\newcommand{\BR}{{\mathbb {R}}}
\newcommand{\BS}{{\mathbb {S}}}
\newcommand{\BT}{{\mathbb {T}}}
\newcommand{\BU}{{\mathbb {U}}}
\newcommand{\BV}{{\mathbb {V}}}
\newcommand{\BW}{{\mathbb {W}}}
\newcommand{\BX}{{\mathbb {X}}}
\newcommand{\BY}{{\mathbb {Y}}}
\newcommand{\BZ}{{\mathbb {Z}}}
\newcommand{\bi}{{\mathbf {i}}}
%%%%%%%%%%%%%%%%%%%%%%%%%%%%%%%%%%%%%%%% Curly Fonts

\newcommand{\CA}{{\mathcal {A}}}
\newcommand{\CB}{{\mathcal {B}}}
\newcommand{\CC}{{\mathcal {C}}}
%\newcommand{\CD}{{\mathcal {D}}}
\newcommand{\CE}{{\mathcal {E}}}
\newcommand{\CF}{{\mathcal {F}}}
\newcommand{\CG}{{\mathcal {G}}}
\newcommand{\CH}{{\mathcal {H}}}
\newcommand{\CI}{{\mathcal {I}}}
\newcommand{\CJ}{{\mathcal {J}}}
\newcommand{\CK}{{\mathcal {K}}}
\newcommand{\CL}{{\mathcal {L}}}
\newcommand{\CM}{{\mathcal {M}}}
\newcommand{\CN}{{\mathcal {N}}}
\newcommand{\CO}{{\mathcal {O}}}
\newcommand{\CP}{{\mathcal {P}}}
\newcommand{\CQ}{{\mathcal {Q}}}
\newcommand{\CR}{{\mathcal {R}}}
\newcommand{\CS}{{\mathcal {S}}}
\newcommand{\CT}{{\mathcal {T}}}
\newcommand{\CU}{{\mathcal {U}}}
\newcommand{\CV}{{\mathcal {V}}}
\newcommand{\CW}{{\mathcal {W}}}
\newcommand{\CX}{{\mathcal {X}}}
\newcommand{\CY}{{\mathcal {Y}}}
\newcommand{\CZ}{{\mathcal {Z}}}

%%%%%%%%%%%%%%%%%%%%%%%%%%%%%%%%%%%%%%%% Roman Fonts

\newcommand{\RA}{{\mathrm {A}}}
%\newcommand{\RB}{{\mathrm {B}}}
\newcommand{\RC}{{\mathrm {C}}}
\newcommand{\RD}{{\mathrm {D}}}
\newcommand{\RE}{{\mathrm {E}}}
\newcommand{\RF}{{\mathrm {F}}}
\newcommand{\RG}{{\mathrm {G}}}
\newcommand{\RH}{{\mathrm {H}}}
\newcommand{\RI}{{\mathrm {I}}}
\newcommand{\RJ}{{\mathrm {J}}}
\newcommand{\RK}{{\mathrm {K}}}
\newcommand{\RL}{{\mathrm {L}}}
\newcommand{\RM}{{\mathrm {M}}}
\newcommand{\RN}{{\mathrm {N}}}
\newcommand{\RO}{{\mathrm {O}}}
\newcommand{\RP}{{\mathrm {P}}}
\newcommand{\RQ}{{\mathrm {Q}}}
\newcommand{\RR}{{\mathrm {R}}}
\newcommand{\RS}{{\mathrm {S}}}
\newcommand{\RT}{{\mathrm {T}}}
\newcommand{\RU}{{\mathrm {U}}}
\newcommand{\RV}{{\mathrm {V}}}
\newcommand{\RW}{{\mathrm {W}}}
\newcommand{\RX}{{\mathrm {X}}}
\newcommand{\RY}{{\mathrm {Y}}}
\newcommand{\RZ}{{\mathrm {Z}}}
\renewcommand{\Re}{{\mathrm {Re}}}
\newcommand{\Ri}{{\mathrm {i}}}
%%%%%%%%%%%%%%%%%%%%%%%%%%%%%%%%%%%%%%%%%%  Gothic Fonts, Lie algebras

\newcommand{\fa}{\mathfrak{a}}
\newcommand{\fb}{\mathfrak{b}}
\newcommand{\fc}{\mathfrak{c}}
\newcommand{\fg}{\mathfrak{g}}
\newcommand{\fh}{\mathfrak{h}}
\newcommand{\fk}{\mathfrak{k}}
\newcommand{\fl}{\mathfrak{l}}
\newcommand{\fm}{\mathfrak{m}}
\newcommand{\fn}{\mathfrak{n}}
\newcommand{\fo}{\mathfrak{o}}
\newcommand{\fp}{\mathfrak{p}}
\newcommand{\fq}{\mathfrak{q}}
\newcommand{\fr}{\mathfrak{r}}
\newcommand{\fs}{\mathfrak{s}}
\newcommand{\ft}{\mathfrak{t}}
\newcommand{\fu}{\mathfrak{u}}
\newcommand{\fv}{\mathfrak{v}}
\newcommand{\fy}{\mathfrak{y}}
\newcommand{\fz}{\mathfrak{z}}
\newcommand{\fgl}{{\mathfrak{gl}}}
\newcommand{\fsl}{{\mathfrak{sl}}}
\newcommand{\fso}{{\mathfrak{so}}}
\newcommand{\fsp}{{\mathfrak{sp}}}
\newcommand{\fsu}{{\mathfrak{su}}}

%%%%%%%%%%%%%%%%%%%%%%%%%%%%%%%%%% Algebraic Groups and Representation

\newcommand{\GL}{{\mathrm{GL}}}
\newcommand{\G}{{\mathrm{G}}}
\newcommand{\U}{{\mathrm{U}}}
\newcommand{\Y}{{\mathrm{Y}}}
%\newcommand{\P}{{\mathrm{P}}}
\newcommand{\A}{{\mathrm{A}}}
\newcommand{\SL}{{\mathrm{SL}}}
\newcommand{\SU}{{\mathrm{SU}}}
\newcommand{\GU}{{\mathrm{GU}}}
\newcommand{\SO}{{\mathrm{SO}}}
\newcommand{\GO}{{\mathrm{GO}}}
\newcommand{\Sp}{{\mathrm{Sp}}}
\newcommand{\GSp}{{\mathrm{GSp}}}
\newcommand{\quo}{\backslash}
\newcommand{\adequo}[1]{#1(\F)\quo #1(\mathds{A})}
\newcommand{\ade}[1]{#1(\mathds{A})}
\newcommand{\Hom}{{\mathrm{Hom}}}
\newcommand{\Bil}{{\mathrm{Bil}}}
\newcommand{\Id}{{\mathrm{Id}}}\newcommand{\id}{{\mathrm{id}}}
\newcommand{\Ind}{{\mathrm{Ind}}}
\newcommand{\Irr}{{\mathrm{Irr}}}
\newcommand{\End}{{\mathrm{End}}}
\newcommand{\Aut}{{\mathrm{Aut}}}
\newcommand{\Inn}{{\mathrm{Inn}}}
\newcommand{\Out}{{\mathrm{Out}}}
\newcommand{\Ad}{{\mathrm{Ad}}}
\newcommand{\ad}{{\mathrm{ad}}}
\newcommand{\Der}{{\mathrm{Der}}}
\newcommand{\Lie}{{\mathrm{Lie}}}
\newcommand{\Ann}{{\mathrm{Ann}}}
\newcommand{\Mat}{{\mathrm{Mat}}}
\newcommand{\sgn}{{\mathrm{sgn}}}
\newcommand{\Rep}{{\mathrm{Rep}}}
\newcommand{\Nil}{{\mathrm{Nil}}}
\newcommand{\ten}{\otimes}
\newcommand{\proten}{\hat{\otimes}}
\newcommand{\Mell}[1]{\mathcal{M}#1}
\newcommand{\Gal}[1]{\Gamma_{#1}}
\newcommand{\Tr}{{\mathrm{Tr}}}
\newcommand{\gr}{{\mathrm{gr}}}
\newcommand{\Sym}{{\mathrm{Sym}}}
\newcommand{\diag}{\mathrm{diag}}

%%%%%%%%%%%%%%%%%%%%%%%%%%%%%%%%%   Math formula

\newcommand{\sbst}{\subseteq}
\newcommand{\norm}[1]{\lVert#1\rVert}
\newcommand{\abs}[1]{\lvert#1\rvert}
\newcommand{\set}[2]{\{#1\,|\,#2\}}
\newcommand{\bigset}[2]{\Biggl\{#1\,\bigg\lvert\,#2\Biggr\}}
\newcommand{\defmap}[5]{
           \begin{equation*}
              \begin{aligned}
                   #1:\quad  & #2 &\longrightarrow &\quad #3 \\
                      \quad  & #4    &\longmapsto  &\quad #5
              \end{aligned}
           \end{equation*}
          }
\newcommand{\mtrtwo}[4]{\begin{pmatrix} #1 &#2 \\#3 &#4 \end{pmatrix}}
\newcommand{\mtrthr}[9]{\begin{pmatrix} #1 &#2 &#3 \\#4 &#5 &#6\\ #7 &#8 &#9 \end{pmatrix}}
\newcommand{\defmtrtwo}[6]{
           \begin{equation*}
               #1 = \bigset{\mtrtwo{#2}{#3}{#4}{#5}}{#6},
           \end{equation*}
           }
\newcommand{\shortexact}[5]{#1\rightarrow #2 \rightarrow #3 \rightarrow #4\rightarrow #5}
\renewcommand{\bar}{\overline}
\renewcommand{\tilde}{\widetilde}
\newcommand{\eps}{\epsilon}

\title[]{Counting Irreducible Representations of General Linear Groups and Unitary Groups}
\author{Qiutong Wang}
\institute{Zhejiang University}
\date{Conference on Relative Langlands Program\\
  Tianyuan Mathematics Reserch Center\\ August 11, 2025}

\begin{document}

\begin{frame}
  \titlepage
\end{frame}

% Uncomment these lines for an automatically generated outline.
%\begin{frame}{Outline}
%  \tableofcontents
%\end{frame}



\begin{frame}{Outline}

\begin{itemize}
  \item Introduce the counting method developed by Dan Barbasch, Jia-Jun Ma, Binyong Sun, and Chen-Bo Zhu in their paper: Special unipotent representations of real classical groups: Counting and reduction.
   \item Show its application in counting the irreducible representations of general linear groups over $\BR, \BC$, or $\BH$, or a real unitary groups.
\end{itemize}
\end{frame}












\begin{frame}{Language for today}
  \begin{itemize}
    \item $\RG$: connected reductive algebraic group defined over $\BR$;
    \item $G$ is a real Lie group together with a Lie group homorphism $\iota: G \to \RG(\BR)$ with open image and finite kernel;
    \item $\fg$, $\fg_0$ are Lie algebras of $\RG(\BC)$, $G$;
    \item $^{a}\fh = \fb/[\fb,\fb]$: the abstract Cartan subalgebra of $\fg$, with root lattice $Q_{\fg}$, weight group $Q^{\fg}$, and analytic weight lattice $Q_{\iota}$ ($Q_{\fg} \subseteq Q_{\iota} \subseteq Q^{\fg} \subseteq {^{a}\fh}^*$);
    \item $W$: the abstract Weyl group of $\fg$ acts on $^{a}\fh^*$;
    \item Let $\nu \in {^{a}\fh}^*$, denote by $\Lambda = \nu + Q_{\iota}$ the translate of analytic lattice $Q_{\iota}$ by $\nu$;
    \item $\Delta$, $\Delta(\Lambda)$ be the root system and integral root system respectively, the integral Weyl group is $W(\Lambda)$.
  \end{itemize}
\end{frame}





\begin{frame}
  \begin{itemize}
    \item $\mathrm{Rep}(G)$ is the category of Casselman-Wallach representations of $G$, whose Grothendieck group (with $\BC$-coefficient) is denoted by $\CK(G)$;
    \item $\mathrm{Irr}(G)$ is the set of isomorphism classes of irreducible objects in $\mathrm{Rep}(G)$;
    \item $\Rep_{\nu}(G)$ is the category of Casselman-Wallach representations of $G$ with generalized infinitesimal character $\nu$;
    \item $\Irr_{\nu}(G)$ is the set of isomorphism classes of irreducible objects in $\mathrm{Rep}_{\nu}(G)$;
    \item For any irreducible representation $\RV \in \Irr(G)$, recall that $I = \mathrm{Ann}(\RV) \subseteq \U(\fg)$ is a primitive ideal, its associated variety $\mathrm{AV}(I) \subseteq \fg^*$ is the closure of a unique nilpotent orbit (Borho-Brylinski and Joseph).
    \item The complex associated variety (annihilator variety) of $\RV$ is $\mathrm{AV}_{\BC}(\RV) = \mathrm{AV}(I) \subseteq \fg^*$. 
  \end{itemize}
\end{frame}




\begin{frame}{Goals}
  \begin{itemize}
  \item The set \(\Irr(G)\) admits a partition according to infinitesimal characters:
  \[
    \Irr(G) = \bigsqcup\limits_{\lambda \in W \backslash {^{a}\mathfrak{h}}^*} \Irr_{\lambda}(G).
  \]
  By early work of Harish-Chandra, each subset \(\Irr_{\lambda}(G)\) is finite.
  
  \item Furthermore, according to complex associated variety, the set \(\Irr_{\nu}(G)\) admits a finer partition:
  \[
    \Irr_{\nu}(G) = \bigsqcup\limits_{\mathcal{O} \in \RG(\mathbb{C}) \backslash \Nil(\mathfrak{g})} \Irr_{\nu}(G; \mathcal{O}).
  \]
  Here $\Irr_{\nu}(G;\CO)$ is the subset of $\Irr_{\nu}(G)$ consists of irreducible representations with complex associated variety $\bar{\CO}$.
\end{itemize}

\medskip

\noindent
\textbf{Goal:} Describe the size of each set \(\Irr_{\lambda}(G; \mathcal{O})\) in terms of some combinatorial data.
\end{frame}









\begin{frame}{Counting Formula}
\begin{block}{Theorem (Barbasch, Ma, Sun, Zhu)}
  \begin{equation*}
        \sharp(\Irr_{\nu}(G;\CO)) \leq \sum\limits_{\sigma \in \Irr(W(\Lambda);\CO)} [1_{W_\nu}:\sigma] \cdot [\sigma:\mathrm{Coh}_{\Lambda}(\CK(G))],
    \end{equation*}
    where $1_{W_\nu}$ denotes the trivial representation of the stabilizer $W_\nu$ of $\nu$ in $W$. The \alert{equality holds} if the Coxeter group $W(\Lambda)$ has no simple factor of type $F_4$, $E_6$, $E_7$, or $E_8$, and $G$ is linear or isomorphic to a real metaplectic group.
\end{block}

 In their paper, they use this formula to count the number of special unipotent representations to construct them via theta correspondence.

\end{frame}




\begin{frame}{Double cells, Special representations}
  
\begin{itemize}
  \item Springer correspondence: $\sigma \in \Irr(W) \leadsto (\CO,\CL)$; If $\CL =1$ we call the corresponding representation a Springer representation;
  \item For the Weyl group $W$, Lusztig define a class of \alert{special representations}, which are Springer representations corresponding to a \alert{special nilpotent orbit}.
  \item Lusztig also defined an equivalence relation on $\Irr(W)$, each equivalence class is called a \alert{double cell}, $\Irr^{sp}(W) \leftrightarrow \{ \textrm{double cells} \}$.
  \item There is a \alert{$j$-induction} (also called the truncated induction) operation, $j_{W(\Lambda)}^{W}: \{ \textrm{special representations $W(\Lambda)$}\} \to \{\textrm{Springer representations of $W$} \}$.
\end{itemize}

\end{frame}





\begin{frame}
  \begin{block}{Definition}
    For a nilpotent orbit $\CO \in \Nil(\fg)$.
    \begin{itemize}
      \item $\Irr^{\mathrm{sp}}(W(\Lambda);\CO) = \set{\sigma \in \Irr(W(\Lambda))}{\textrm{$\sigma$ is special}, j_{W(\Lambda)}^{W}(\sigma) = \sigma_{\CO}}$;
      \item $\Irr(W(\Lambda);\CO) = \set{\sigma \in \Irr(W(\Lambda))}{\textrm{exist $\sigma_{0} \in \Irr^{\mathrm{sp}}(W(\Lambda);\CO)$}, \sigma \approx \sigma_{0}, }$
    \end{itemize}
  \end{block}
  $\Irr(W(\Lambda);\CO)$ is a union of several double cells.
\end{frame}



\begin{frame}{Coherent continuation representations}
  $\CR_{\mathrm{hol}}(\RG(\BC))$: Grothendieck ring of finite-dimensional holomorphic representations of $\RG(\BC)$.\par
  $\CK(G)$ has a $\CR_{\mathrm{hol}}(\RG(\BC))$ module structure via tensor product.
  \begin{block}{Coherent family}
    Let $\Lambda = \nu + Q_{\iota} \subseteq {^{a}\fh}^*$, a \alert{$\Lambda$-coherent family}  is a map
   $$\Phi: \Lambda \to \CK(G),$$
   such that:
   \begin{itemize}
      \item For any $\mu \in \Lambda$, $\Phi(\mu) \in \CK_{\mu}(G)$;
      \item For any $F \in \CR_{\mathrm{hol}}(\RG(\BC))$ and $\mu \in \Lambda$, $F \cdot (\Phi(\mu)) = \sum_{\lambda \in \Delta(F)} \Phi(\mu + \lambda)$ (where $\Delta(F)$ is the set of weights of $F$ counted multiplicity).
   \end{itemize}
  \end{block}
\end{frame}






\begin{frame}
  \begin{block}{Coherent continuation representation}
  Let $\mathrm{Coh}_{\Lambda}(\CK(G))$ denote the complex vector space of all coherent families on $\Lambda$. It is a representation of $W(\Lambda)$ under the action
  \[(w \cdot \Psi)(\nu) = \Psi(w^{-1}\nu),\]
  for any $w \in W(\Lambda)$, $\Psi \in \mathrm{Coh}_{\Lambda}(\CK(G))$, $\nu \in \Lambda$.
  \end{block}
     For any $\mu \in \Lambda$ we have the evaluation map \defmap{\mathrm{ev}}{\mathrm{Coh}_{\Lambda}(\CK(G))}{\CK_{\mu}(G)}{\Psi}{\Psi(\mu)}
  \begin{block}{Theorem (Schmid, Zuckerman)}
    $\mathrm{ev}$ is surjective for each $\mu \in \Lambda$, and bijective when $\mu$ is regular.
  \end{block}

\end{frame}






\begin{frame}{Harish-Chandra cells}
  \begin{block}{Theorem}
    Suppose $\nu \in {^{a}\fh}^*$ dominant, $M \in \CK_{\nu}(G)$ is an irreducible representation. Then there exist a unique coherent family $\bar{\Psi}$ characterised by the following properties:
    \begin{itemize}
      \item $\bar{\Psi}(\nu) = M$;
      \item If $\mu$ is dominant, then $\bar{\Psi}(\mu)$ is irreducible or zero.
    \end{itemize}
  \end{block}
   There is a basis $\CB = \{\bar{\Psi}_i\}$ of $\mathrm{Coh}_{\Lambda}(\CK(G))$ such that for any regular dominant $\mu$, $\bar{\Psi}_i(\mu)$ is an irreducible representation with infinitesimal character $\mu$ (there is also a basis $\Psi_i$ of standard modules).\par
   We view $\mathrm{Coh}_{\Lambda}(\CK(G))$ as a \textcolor{red}{basal representation} with basal elements $\CB$.\par
   We can also define basal subrepresentations, which are subrepresentations of $\mathrm{Coh}_{\Lambda}(\CK(G))$ spanned by a subset of $\CB$. 
\end{frame}






\begin{frame}
  \begin{itemize}
    \item For any subset $\CS$ of $\mathrm{Coh}_{\Lambda}(\CK(G))$, denote by $\left \langle \CS \right \rangle$ the minimal basal subrepresentation containing $\CS$.
    \item Define an equivalence relation on $\CB$ by: 
    $\bar{\Psi}_i \approx  \bar{\Psi}_j$ if and only if $\left \langle \bar{\Psi}_i \right \rangle = \left \langle \bar{\Psi}_j \right \rangle$.
    \item The equivalence classes of $\CB$ under this relation are called Harish-Chandra cells.
  \end{itemize}
  
  
  
  \begin{block}{Cell representations}
    Let $\CC$ be a cell in $\CB$ and put $\bar{\CC} = \left \langle \CC \right \rangle \cap \CB$. Define the cell representation attached to $\CC$ by
    \[ \mathrm{Coh}_{\Lambda}(\CK(G))(\CC) := \left \langle \bar{\CC} \right \rangle  / \left \langle \bar{\CC} \backslash \CC \right \rangle\] 
  \end{block}

 It is easy to see that the image of $\CC$ under this quotient form a basis of $\mathrm{Coh}_{\Lambda}(\CK(G))(\CC)$.
  
\end{frame}






\begin{frame}{Kazhdan-Lusztig cells}
  \begin{itemize}
    \item Consider the category $\Rep(\fg,\fb)$ of finite generated $\U(\fg)$-module which is locally finite over $\U(\fb)$, denote its Grothendieck group by $\CK(\fg,\fb)$;
    \item The Grothendieck ring $\CR(\fg,Q_{\iota})$ ($\simeq \CR_{\mathrm{hol}}(\RG(\BC))$) acts on $\CK(\fg,\fb)$ via tensor product, so we can also define a coherent family $\mathrm{Coh}_{\Lambda}(\CK(\fg,\fb))$;
    \item It has two basis $\set{\Psi_w}{w \in W}$, $\set{\bar{\Psi}_{w}}{w \in W}$ where $\Psi_w(\mu) = \RM(w\mu)$, $\bar{\Psi}_{w}(\mu) = \RL(w\mu)$ for regular dominant $\mu \in {^{a}\fh}^*$;
    \item Define the $W \times W(\Lambda)$ action explicitly by \[(w_1,w_2) \cdot \Psi_w = \Psi_{w_1 w w_2^{-1}} \ \textrm{for all $w_1 \in W, w_2 \in W(\Lambda)$};\]
    \item We view $\mathrm{Coh}_{\Lambda}(\CK(\fg,\fb))$ as a basal representation with basal elements $\{\bar{\Psi}_w\}$.
    
  \end{itemize}
  
\end{frame}



\begin{frame}
\begin{itemize}
  \item The cells of $\mathrm{Coh}_{\Lambda}(\CK(\fg,\fb))$ under $W \times W(\Lambda)$ action is called a \alert{Kazhdan-Lusztig cells} (two-side cells);
  \item Each cell representation $\mathrm{Coh}_{\Lambda}(\CK(\fg,\fb))(\CC)$ contains a unique special representation of $W(\Lambda)$ called $\sigma_{\CC}$, this induce a bijection between Kazhdan-Lusztig cells and special representations of $W(\Lambda)$;
  \item As a representation of $W \times W(\Lambda)$, \[\mathrm{Coh}_{\Lambda}(\CK(\fg,\fb))(\CC) \simeq \sum\limits_{\textrm{$\sigma$ in the double cell contain $\sigma_{\CC}$}}(\Ind_{W(\Lambda)}^{W}\sigma) \otimes \sigma.\]
\end{itemize}


  
  
\end{frame}





\begin{frame}{Comparison of HC cells and KL cells}
  \begin{itemize}
    \item For every Harish-Chandra cell we can attach a Kazhdan-Lusztig cell to it (via comparing their annihilator ideals); furthermore, we have a map $\{\textrm{Harish-Chandra  cells}\} \to \Irr^{sp}(W(\Lambda))$, $\CC \mapsto \sigma_{\CC}$.
  \item It is a well-known fact that every representation in a Harish-Chandra cell $\CC$ has the \alert{same complex associated variety} $\bar{\CO_{\CC}}$, where $\CO_{\CC}$ is the nilpotent orbit corresponding to $j_{W(\Lambda)}^{W}\sigma_{\CC}$.
  \end{itemize}
  \begin{block}{Conjecture}
    The set \set{$\sigma \in \Irr(W(\Lambda))$}{ \textrm{$\sigma$ occurs in $\mathrm{Coh}_{\Lambda}(\CK(G))(\CC)$}}\\[3pt] is contained in the double cell containing the special representation $\sigma_{\CC}$.
  \end{block}
   BMSZ proved this conjecture holds under some technical assumptions ($W(\Lambda)$ has no simple factor of type $F_4$, $E_6$, $E_7$, or $E_8$, and $G$ is linear or isomorphic to a real metaplectic group).
\end{frame}



\begin{frame}{BMSZ's proof of the counting formula}
  If $S$ is a Zariski closed $\RG(\BC)$-stable subset of $\Nil(\fg)$, then
  \begin{align*}
    \sharp(\Irr_{\nu,S}(G)) &= \dim \CK_{\nu,S}(G) = \dim \mathrm{Coh}_{\Lambda}(\CK_{S}(G))_{W_\nu} \\
    & = [1_{W_{\nu}}:\mathrm{Coh}_{\Lambda}(\CK_{S}(G))] \\
    & = \sum_{\sigma \in \Irr(W(\Lambda))} [1_{W_\nu}:\sigma] \cdot [\sigma:\mathrm{Coh}_{\Lambda}(\CK_{\RS}(G))]\\
    & \xlongequal{\alert{Conjecture}} \sum_{\sigma \in \Irr_{S}(W(\Lambda))} [1_{W_\nu}:\sigma] \cdot [\sigma:\mathrm{Coh}_{\Lambda}(\CK(G))].
  \end{align*}
\end{frame}








\begin{frame}{Classical groups considered}

\begin{center}
   \begin{tabular}{c|c|c}
      \toprule
      Label $\star $ & Classical Lie Group $G$ & Complex Lie Group $\RG(\BC)$    \\
      \midrule
      $A^{\BR}$      & $\GL_n(\BR)$          & $\GL_n(\BC)$                   \\
      $A^{\BH}$      & $\GL_{\frac{n}{2}}(\BH)$ \  ($n$ is even)      & $\GL_n(\BC)$                   \\
      $A^{\BC}$      & $\GL_n(\BC)$          & $\GL_n(\BC) \times \GL_n(\BC)$ \\
      $A$            & $\U(p,q)$              & $\GL_n(\BC)$ \ $(n = p + q)$                   \\
      \bottomrule
   \end{tabular}
\end{center}

 Identification:

\[{^{a}\fh}^* = \left\{
   \begin{aligned}
       & \BC^n, & \textrm{if $\star \in \{A^\BR,A^\BH,A\}$};\\
       & \BC^n \times \BC^n, & \textrm{if $\star = A^\BC$}.
   \end{aligned}
   \right.
\]
\[
    Q_{\iota} = \left\{
    \begin{aligned}
        & \BZ^n \subseteq \BC^n = {^{a}\fh^*}, \ &\textrm{if $\star \in \{A^\BR, A^\BH,A\}$};\\
        & \BZ^n \times \BZ^n \subseteq \BC^n \times \BC^n = {^{a}\fh^*}, \ &\textrm{if $\star = A^\BC$},
   \end{aligned}
   \right.
\]
\end{frame}






\begin{frame}{Combinatorial notions}
  \begin{block}{Painted Young Diagram (type $A^{\BR}$)}
    A painting on a Young diagram $\iota$ of type $A^{\BR}$ is a map (we place a symbol in each box)
   $$\CP : \mathrm{Box}(\iota) \to \{ \bullet, c ,d \}$$
   With the following properties

   \begin{itemize}
      \item if we remove the boxes painted with $\{d\}, \{c,d\}$, the remainder still constitutes a Young diagram;
      \item every column of $\iota$ has at most one box painted with $c$, and has at most one box painted with $d$;
      \item every row of $\iota$ has an even number of boxes painted with $\bullet$.
   \end{itemize}
    A painted Young diagram is a pair $(\iota, \CP)$ consisting of a Young diagram $\iota$ and a painting $\CP$ on $\iota$. Denote by $\mathrm{P}_{A^{\BR}}(\iota)$ the set of paintings on $\iota$ of type $A^{\BR}$.
  \end{block}
  
\end{frame}

\begin{frame}
  \begin{block}{Example}
    The following represents a painted Young diagram.
    \[
    \begin{ytableau}
        \bullet & \bullet & c & d  \\
        \bullet & \bullet & d \\
        c \\
        d
    \end{ytableau}
    \]
     Each of the following does not represent a painted Young diagram.
    \[
    \begin{ytableau}
      d & c
    \end{ytableau}
    \qquad
    \begin{ytableau}
      c & d\\
      c
    \end{ytableau}
    \]
  \end{block}
  
\end{frame}






\begin{frame}
  \begin{block}{Assigned Young Diagram}
    For a Young diagram $\iota$, and a partition $[d_1, \cdots, d_k]$ of $|\iota|$.  An assignment of type $[d_1,d_2, \cdots, d_N]$ on $\iota$ is a map
   $$\CQ: \mathrm{Box}(\iota) \to \{1,2, \cdots,N\} $$
   With the following properties

   \begin{itemize}
      \item for each $i \in \{1,2,\cdots,N\}$, the preimage $\CP^{-1}(i)$ has exactly $d_i$ elements;
      \item for each $1 \leq n \leq N$, if we remove the boxes assigned with $\{n+1, \cdots, N\}$, the reminder still constitutes a Young diagram;
      \item each positive integer occurs at most once in each column.
   \end{itemize}
    An assigned Young diagram of type $[d_1,d_2, \cdots, d_N]$ is a pair $(\iota,\CQ)$ consisting of a Young diagram $\iota$ and an assignment $\CQ$ of type $[d_1,d_2, \cdots, d_N]$ on $\iota$. Denote by $\RA_{[d_1,d_2,\cdots,d_N]}(\iota)$ the set of all assignments on $\iota$ of type $[d_1,d_2,\cdots,d_N]$.
  \end{block}
\end{frame}



\begin{frame}{Counting result for $\GL_n(\BR)$: The integral case}
  $G = \GL_n(\BR)$ ($n \in \BN$). $\CO$ is a nilpotent orbits in $\fg$, denote the corresponding Young diagram by $\iota(\CO)$.\par
    If $\nu \in {^{a}\fh^*} = \BC^n$ is integral, its coordinates can be permuted such that 
        \[
        \nu = (\underbrace{\lambda_1, \cdots, \lambda_1}_{d_1}, \underbrace{\lambda_2, \cdots, \lambda_2}_{d_2}, \cdots, \underbrace{\lambda_k, \cdots, \lambda_k}_{d_k}) \in \BC^n,
        \]
        where $[d_1, d_2, \cdots , d_k]$ is a partition of $n$, and the $\lambda_i \in \BC$ satisfy the condition $\lambda_i - \lambda_j \in \BZ \setminus \{0\}$ for any $i \neq j$.
  \begin{block}{Theorem}
        \begin{equation*}
            \sharp(\Irr_\nu(\GL_{n}(\BR);\CO)) = \sharp\left(\mathrm{P}_{A^{\BR}}(\iota(\CO))\right)\cdot \sharp\left(\RA_{[d_1,\cdots,d_k]}(\iota(\CO))\right).
        \end{equation*}
  \end{block}
\end{frame}





\begin{frame}{Sketch of the calculation}
  Coherent continuation representations can be explicitly computed via the Hecke algebra module structure described by Lusztig and Vogan.
  \begin{equation*}
    \mathrm{Coh}_{\Lambda}(\CK(\GL_{n}(\BR))) = \bigoplus_{2r + i \leq n} \Ind _{\mathrm{W}_{r} \times \mathrm{S}_{i} \times \mathrm{S}_{n-2r-i}}^{\mathrm{S}_{n}} \epsilon \otimes 1 \otimes 1.
  \end{equation*}
   The computation makes essential use of \alert{\emph{Pieri's rule}} and Frobenius reciprocity.
  \begin{align*}
    \sharp(\Irr_\nu(\GL_{n}(\BR);\CO)) &= [1_{W_\nu}:\sigma_{\CO}] \cdot [\sigma_{\CO}:\mathrm{Coh}_{\Lambda}(\CK(\GL_{n}(\BR)))]\\
    &= [\sigma_{\CO}:\Ind_{W_{\nu}}^{W}1_{W_{\nu}}] \cdot [\sigma_{\CO}:\mathrm{Coh}_{\Lambda}(\CK(\GL_{n}(\BR)))]\\
    & =  \sharp\left(\RA_{[d_1,\cdots,d_k]}(\iota(\CO))\right) \cdot \sharp\left(\mathrm{P}_{A^{\BR}}(\iota(\CO))\right).
  \end{align*}
\end{frame}






\begin{frame}{Example: Minimal representations}

Let \( \nu \in {^a\mathfrak{h}}^* \) be a regular integral infinitesimal character, and let \( \mathcal{O}_{\mathrm{min}} \) denote the \alert{minimal nilpotent orbit}.

\medskip

There are \( n-1 \) assignments of type \( [\underbrace{1, 1, \dots, 1}_n] \) on it, given by
\[
\begin{ytableau}
    1 & \scriptstyle i+1 \\
    2 \\
    \vdots \\
    i \\
    \scriptstyle i+2 \\
    \vdots \\
    n
\end{ytableau}
\quad \text{for } 1 \leq i \leq n-1.
\]

\medskip

 Since there are 4 distinct paintings on $\iota_{\CO}$, we obtain exactly \alert{4(n-1) minimal representations}   with the fixed infinitesimal character \( \nu \).

\end{frame}






\begin{frame}{Non-integral case}
  For an arbitrary $\nu \in {^{a}\fh}^*$, its coordinates can be permuted such that 
        \[
        \nu = (\blam_1, \cdots, \blam_r) \in \BC^n,
        \]
        \[
        \blam_i = (\underbrace{\lambda_{i,1}, \cdots, \lambda_{i,1}}_{d_{i,1}}, \cdots, \underbrace{\lambda_{i,k_i}, \cdots, \lambda_{i,k_i}}_{d_{i,k_i}}) \in \BC^{e_i} \ \ (e_i \geq 1),
        \]  
        where each $\blam_i$ is integral but $(\blam_i,\blam_j) \in \BC^{e_i+e_j}$ is not integral for any $i \neq j$, $[d_{i,1}, \cdots d_{i,k_i} ]$ is a partition of $e_i$, and the condition $\lambda_{i,p} - \lambda_{i,q} \in \BZ \setminus \{0\}$ holds for any $p \neq q$.
  \begin{block}{Theorem}
    
        \begin{equation*}
            \sharp(\Irr_{\nu}(G;\CO)) = \sum_{\substack{(\iota_1,\cdots,\iota_r) \in \mathrm{YD}_{e_1} \times \cdots \times \mathrm{YD}_{e_r} \\ \iota_1 \mathop{\sqcup}\limits^r \iota_2 \cdots \mathop{\sqcup}\limits^r  \iota_r = \iota(\CO) }}\prod_{i=1}^{r}\sharp(\Irr_{\blam_i}(\GL_{e_i}(\BR);\CO_{\iota_i}))
        \end{equation*}
  \end{block}
\end{frame}






\begin{frame}{Key observation}
  There are natural isomorphism of $W(\Lambda)$-representations:
  \begin{align*}
    \mathrm{Coh}_{\Lambda_1}(\CK(\GL_{e_1}(\BR))) \otimes \cdots \otimes \mathrm{Coh}_{\Lambda_r}(\CK(\GL_{e_r}(\BR))) &\to \mathrm{Coh}_{\Lambda}(\CK(\GL_{n}(\BR)))\\
    \Psi_1 \otimes \cdots \otimes \Psi_r & \mapsto \Psi
  \end{align*}
  where $\Psi(\mu) = \Ind_{P}^{\GL_{n}(\BR)}\Psi_1(\mu) \otimes \cdots \otimes \Psi_r(\mu)$ is also a standard module for regular dominant $\mu \in {^{a}\fh}^*$.\par
  Actually, this isomorphism also takes irreducible objects to irreducible objects.
\end{frame}


\begin{frame}{Combinatorial data}
  \begin{block}{Painted Young diagram (type $A$)}
    A painting on a Young diagram $\iota$ of type $A$ is a map (we place a symbol in each box)
   $$\CP : \mathrm{Box}(\iota) \to \{ \bullet, s ,r \}$$
   With the following properties

   \begin{itemize}
      \item if we remove the boxes painted with $\{s\}, \{s,r\}$, the remainder still constitutes a Young diagram;
      \item every row of $\iota$ has at most one box painted with $c$, and has at most one box painted with $d$;
      \item every row of $\iota$ has an even number of boxes painted with $\bullet$.
   \end{itemize}
    A painted Young diagram is a pair $(\iota, \CP)$ consisting of a Young diagram $\iota$ and a painting $\CP$ on $\iota$. Denote by $\mathrm{P}_{A}(\iota)$ the set of paintings on $\iota$ of type $A$.
    
  \end{block}
\end{frame}






\begin{frame}
  Let $\iota$ be a Young diagram and $\CP$ be a painting on $\iota$ of type $A$. Define the signature of $\CP$ to be the pair of non-negative integers
   \begin{equation*}
    \left(p_{\CP}, q_{\CP}\right) := \left(\frac{\sharp(\CP^{-1}(\bullet))}{2} + \sharp(\CP^{-1}(s)), \frac{\sharp(\CP^{-1}(\bullet))}{2}+ \sharp(\CP^{-1}(r))\right),
   \end{equation*}
   for every $p,q \in \BN$ such that $p + q = |\iota|$, we define
   \begin{equation*}
        \mathrm{P}_{A}^{p,q}(\iota) := \set{\CP \in \mathrm{P}_{A}(\iota)}{(p_{\CP},q_{\CP}) = (p,q)}.       
   \end{equation*}
\end{frame}






\begin{frame}{Counting result for $\U(p,q)$: The integral case}
  $G = \U(p,q)$ ($p, q \in \BN$). $\CO$ is a nilpotent orbit in $\fg$, denote the corresponding Young diagram by $\iota(\CO)$.\par
  If $\nu \in {^a\fh^*} = \BC^n$ ($n = p+q$) is integral, that is, the differences of its coordinates are integral, then its coordinates can be permuted such that 
  \[ 
    \nu =  (\underbrace{\lambda_1, \cdots, \lambda_1}_{d_1}, \underbrace{\lambda_2, \cdots, \lambda_2}_{d_2}, \cdots, \underbrace{\lambda_k, \cdots, \lambda_k}_{d_k} ) \in \BC^n, 
  \]
  where $[d_1, d_2, \cdots, d_k]$ is a partition of $n$, and the $\lambda_i \in \BC$ satisfy the condition $\lambda_i - \lambda_j \in \BZ \backslash \{0\}$ for any $i \neq j$.
 
\end{frame}





\begin{frame}
   \begin{block}{Theorem}
        If $\lambda_1 \in \frac{n-1}{2} + \BZ$, then
        \begin{equation*}
            \sharp(\Irr_{\nu}(\U(p,q);\CO)) = \sharp\left(\mathrm{P}_{A}^{p,q}(\iota(\CO))\right) \cdot \sharp\left(\mathrm{A}_{[d_1,\cdots,d_k]}(\iota(\CO))\right).
        \end{equation*}
        If $\lambda_1 \in \frac{n}{2} + \BZ$, then
        \begin{equation*}
            \sharp(\Irr_{\nu}(\U(p,q);\CO)) = \sharp\left(\mathrm{A}_{[d_1,\cdots,d_k]}(\iota(\CO))\right) \cdot \delta_{p,q}.
        \end{equation*}
        Otherwise, $\sharp(\Irr_{\nu}(\U(p,q);\CO)) = 0$.
  \end{block}
\end{frame}






\begin{frame}{Example: Generic representations}

Let \( G = \U(n,n) \), and let \( \CO_{\mathrm{prin}} \) denote the principal nilpotent orbit.

\medskip

If the coordinates of \( \nu \) are all \alert{half-integers}, there is only one assignment on \( \iota(\CO_{\mathrm{prin}}) \), and exactly two paintings:
\[
\begin{ytableau}
    \bullet & \bullet & \cdots & \bullet & \bullet & \bullet
\end{ytableau}
\quad\text{and}\quad
\begin{ytableau}
    \bullet & \bullet & \cdots & \bullet & s & r
\end{ytableau}
\]
Hence, there are exactly \alert{2 generic representations} with this infinitesimal character.

\medskip

 If the coordinates of \( \nu \) are all \alert{integers}, then there is only one assignment. Therefore, there is only \alert{1 generic representation} with this infinitesimal character.

\end{frame}






\begin{frame}[plain]

  \begin{center}

        \font\endfont = cmss10 at 25.40mm
        \color{Brown}
        \endfont 
        \baselineskip 20.0mm

        Thank you

  \end{center}

\end{frame}






\end{document}