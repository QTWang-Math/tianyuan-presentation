\documentclass{beamer}
%
% Choose how your presentation looks.
%
% For more themes, color themes and font themes, see:
% http://deic.uab.es/~iblanes/beamer_gallery/index_by_theme.html
%
\mode<presentation>
{
  \usetheme{Warsaw}      % or try Darmstadt, Madrid, Warsaw, ...
  \usecolortheme{default} % or try albatross, beaver, crane, ...
  \usefonttheme{default}  % or try serif, structurebold, ...
  \setbeamertemplate{navigation symbols}{}
  \setbeamertemplate{caption}[numbered]
} 

\usepackage[english]{babel}
\usepackage[utf8x]{inputenc}
\usepackage{amsmath, amssymb, amsthm}
\usepackage{ytableau}
\usepackage{graphicx}
\usepackage{tikz}
\usepackage{tikz-cd}
\usepackage{xcolor}
\usepackage{extarrows}

%%%%%%%%%%%%%%%%%%%%%%%%%%%%%%%%%%%%%%%%%%%%%% Theorem style

\newtheorem{thm}{Theorem}[section]
\newtheorem{cor}[thm]{Corollary}
\newtheorem{prop}[thm]{Proposition}
\newtheorem{lem}[thm]{Lemma}
\newtheorem{assump}[thm]{Assumption}
\newtheorem{conj}[thm]{Conjecture}
\newtheorem{rk}[thm]{Remark}
\newtheorem{question}[thm]{Question}
\newtheorem{defn}[thm]{Definition}
\newtheorem{con}[thm]{Construction}
\newtheorem{examp}[thm]{Example}
\newtheorem{notn}[thm]{Notation}
\newtheorem{exer}[thm]{Exercise}

%%%%%%%%%%%%%%%%%%%%%%%%%%%%% Bold Fonts
\newcommand{\blam}{{\boldsymbol{\lambda}}}
\newcommand{\bnu}{{\boldsymbol{\nu}}}
\newcommand{\br}{{\mathbf{r}}}
\newcommand{\bc}{{\mathbf{c}}}
\newcommand{\BA}{{\mathbb {A}}}
\newcommand{\BB}{{\mathbb {B}}}
\newcommand{\BC}{{\mathbb {C}}}
\newcommand{\BD}{{\mathbb {D}}}
\newcommand{\BE}{{\mathbb {E}}}
\newcommand{\BF}{{\mathbb {F}}}
\newcommand{\BG}{{\mathbb {G}}}
\newcommand{\BH}{{\mathbb {H}}}
\newcommand{\BI}{{\mathbb {I}}}
\newcommand{\BJ}{{\mathbb {J}}}
\newcommand{\BK}{{\mathbb {U}}}
\newcommand{\BL}{{\mathbb {L}}}
\newcommand{\BM}{{\mathbb {M}}}
\newcommand{\BN}{{\mathbb {N}}}
\newcommand{\BO}{{\mathbb {O}}}
\newcommand{\BP}{{\mathbb {P}}}
\newcommand{\BQ}{{\mathbb {Q}}}
\newcommand{\BR}{{\mathbb {R}}}
\newcommand{\BS}{{\mathbb {S}}}
\newcommand{\BT}{{\mathbb {T}}}
\newcommand{\BU}{{\mathbb {U}}}
\newcommand{\BV}{{\mathbb {V}}}
\newcommand{\BW}{{\mathbb {W}}}
\newcommand{\BX}{{\mathbb {X}}}
\newcommand{\BY}{{\mathbb {Y}}}
\newcommand{\BZ}{{\mathbb {Z}}}
\newcommand{\bi}{{\mathbf {i}}}
%%%%%%%%%%%%%%%%%%%%%%%%%%%%%%%%%%%%%%%% Curly Fonts

\newcommand{\CA}{{\mathcal {A}}}
\newcommand{\CB}{{\mathcal {B}}}
\newcommand{\CC}{{\mathcal {C}}}
%\newcommand{\CD}{{\mathcal {D}}}
\newcommand{\CE}{{\mathcal {E}}}
\newcommand{\CF}{{\mathcal {F}}}
\newcommand{\CG}{{\mathcal {G}}}
\newcommand{\CH}{{\mathcal {H}}}
\newcommand{\CI}{{\mathcal {I}}}
\newcommand{\CJ}{{\mathcal {J}}}
\newcommand{\CK}{{\mathcal {K}}}
\newcommand{\CL}{{\mathcal {L}}}
\newcommand{\CM}{{\mathcal {M}}}
\newcommand{\CN}{{\mathcal {N}}}
\newcommand{\CO}{{\mathcal {O}}}
\newcommand{\CP}{{\mathcal {P}}}
\newcommand{\CQ}{{\mathcal {Q}}}
\newcommand{\CR}{{\mathcal {R}}}
\newcommand{\CS}{{\mathcal {S}}}
\newcommand{\CT}{{\mathcal {T}}}
\newcommand{\CU}{{\mathcal {U}}}
\newcommand{\CV}{{\mathcal {V}}}
\newcommand{\CW}{{\mathcal {W}}}
\newcommand{\CX}{{\mathcal {X}}}
\newcommand{\CY}{{\mathcal {Y}}}
\newcommand{\CZ}{{\mathcal {Z}}}

%%%%%%%%%%%%%%%%%%%%%%%%%%%%%%%%%%%%%%%% Roman Fonts

\newcommand{\RA}{{\mathrm {A}}}
%\newcommand{\RB}{{\mathrm {B}}}
\newcommand{\RC}{{\mathrm {C}}}
\newcommand{\RD}{{\mathrm {D}}}
\newcommand{\RE}{{\mathrm {E}}}
\newcommand{\RF}{{\mathrm {F}}}
\newcommand{\RG}{{\mathrm {G}}}
\newcommand{\RH}{{\mathrm {H}}}
\newcommand{\RI}{{\mathrm {I}}}
\newcommand{\RJ}{{\mathrm {J}}}
\newcommand{\RK}{{\mathrm {K}}}
\newcommand{\RL}{{\mathrm {L}}}
\newcommand{\RM}{{\mathrm {M}}}
\newcommand{\RN}{{\mathrm {N}}}
\newcommand{\RO}{{\mathrm {O}}}
\newcommand{\RP}{{\mathrm {P}}}
\newcommand{\RQ}{{\mathrm {Q}}}
\newcommand{\RR}{{\mathrm {R}}}
\newcommand{\RS}{{\mathrm {S}}}
\newcommand{\RT}{{\mathrm {T}}}
\newcommand{\RU}{{\mathrm {U}}}
\newcommand{\RV}{{\mathrm {V}}}
\newcommand{\RW}{{\mathrm {W}}}
\newcommand{\RX}{{\mathrm {X}}}
\newcommand{\RY}{{\mathrm {Y}}}
\newcommand{\RZ}{{\mathrm {Z}}}
\renewcommand{\Re}{{\mathrm {Re}}}
\newcommand{\Ri}{{\mathrm {i}}}
%%%%%%%%%%%%%%%%%%%%%%%%%%%%%%%%%%%%%%%%%%  Gothic Fonts, Lie algebras

\newcommand{\fa}{\mathfrak{a}}
\newcommand{\fb}{\mathfrak{b}}
\newcommand{\fc}{\mathfrak{c}}
\newcommand{\fg}{\mathfrak{g}}
\newcommand{\fh}{\mathfrak{h}}
\newcommand{\fk}{\mathfrak{k}}
\newcommand{\fl}{\mathfrak{l}}
\newcommand{\fm}{\mathfrak{m}}
\newcommand{\fn}{\mathfrak{n}}
\newcommand{\fo}{\mathfrak{o}}
\newcommand{\fp}{\mathfrak{p}}
\newcommand{\fq}{\mathfrak{q}}
\newcommand{\fr}{\mathfrak{r}}
\newcommand{\fs}{\mathfrak{s}}
\newcommand{\ft}{\mathfrak{t}}
\newcommand{\fu}{\mathfrak{u}}
\newcommand{\fv}{\mathfrak{v}}
\newcommand{\fy}{\mathfrak{y}}
\newcommand{\fz}{\mathfrak{z}}
\newcommand{\fgl}{{\mathfrak{gl}}}
\newcommand{\fsl}{{\mathfrak{sl}}}
\newcommand{\fso}{{\mathfrak{so}}}
\newcommand{\fsp}{{\mathfrak{sp}}}
\newcommand{\fsu}{{\mathfrak{su}}}

%%%%%%%%%%%%%%%%%%%%%%%%%%%%%%%%%% Algebraic Groups and Representation

\newcommand{\GL}{{\mathrm{GL}}}
\newcommand{\G}{{\mathrm{G}}}
\newcommand{\U}{{\mathrm{U}}}
\newcommand{\Y}{{\mathrm{Y}}}
%\newcommand{\P}{{\mathrm{P}}}
\newcommand{\A}{{\mathrm{A}}}
\newcommand{\SL}{{\mathrm{SL}}}
\newcommand{\SU}{{\mathrm{SU}}}
\newcommand{\GU}{{\mathrm{GU}}}
\newcommand{\SO}{{\mathrm{SO}}}
\newcommand{\GO}{{\mathrm{GO}}}
\newcommand{\Sp}{{\mathrm{Sp}}}
\newcommand{\GSp}{{\mathrm{GSp}}}
\newcommand{\quo}{\backslash}
\newcommand{\adequo}[1]{#1(\F)\quo #1(\mathds{A})}
\newcommand{\ade}[1]{#1(\mathds{A})}
\newcommand{\Hom}{{\mathrm{Hom}}}
\newcommand{\Bil}{{\mathrm{Bil}}}
\newcommand{\Id}{{\mathrm{Id}}}\newcommand{\id}{{\mathrm{id}}}
\newcommand{\Ind}{{\mathrm{Ind}}}
\newcommand{\Irr}{{\mathrm{Irr}}}
\newcommand{\End}{{\mathrm{End}}}
\newcommand{\Aut}{{\mathrm{Aut}}}
\newcommand{\Inn}{{\mathrm{Inn}}}
\newcommand{\Out}{{\mathrm{Out}}}
\newcommand{\Ad}{{\mathrm{Ad}}}
\newcommand{\ad}{{\mathrm{ad}}}
\newcommand{\Der}{{\mathrm{Der}}}
\newcommand{\Lie}{{\mathrm{Lie}}}
\newcommand{\Ann}{{\mathrm{Ann}}}
\newcommand{\Mat}{{\mathrm{Mat}}}
\newcommand{\sgn}{{\mathrm{sgn}}}
\newcommand{\Rep}{{\mathrm{Rep}}}
\newcommand{\Nil}{{\mathrm{Nil}}}
\newcommand{\ten}{\otimes}
\newcommand{\proten}{\hat{\otimes}}
\newcommand{\Mell}[1]{\mathcal{M}#1}
\newcommand{\Gal}[1]{\Gamma_{#1}}
\newcommand{\Tr}{{\mathrm{Tr}}}
\newcommand{\gr}{{\mathrm{gr}}}
\newcommand{\Sym}{{\mathrm{Sym}}}
\newcommand{\diag}{\mathrm{diag}}

%%%%%%%%%%%%%%%%%%%%%%%%%%%%%%%%%   Math formula

\newcommand{\sbst}{\subseteq}
\newcommand{\norm}[1]{\lVert#1\rVert}
\newcommand{\abs}[1]{\lvert#1\rvert}
\newcommand{\set}[2]{\{#1\,|\,#2\}}
\newcommand{\bigset}[2]{\Biggl\{#1\,\bigg\lvert\,#2\Biggr\}}
\newcommand{\defmap}[5]{
           \begin{equation*}
              \begin{aligned}
                   #1:\quad  & #2 &\longrightarrow &\quad #3 \\
                      \quad  & #4    &\longmapsto  &\quad #5
              \end{aligned}
           \end{equation*}
          }
\newcommand{\mtrtwo}[4]{\begin{pmatrix} #1 &#2 \\#3 &#4 \end{pmatrix}}
\newcommand{\mtrthr}[9]{\begin{pmatrix} #1 &#2 &#3 \\#4 &#5 &#6\\ #7 &#8 &#9 \end{pmatrix}}
\newcommand{\defmtrtwo}[6]{
           \begin{equation*}
               #1 = \bigset{\mtrtwo{#2}{#3}{#4}{#5}}{#6},
           \end{equation*}
           }
\newcommand{\shortexact}[5]{#1\rightarrow #2 \rightarrow #3 \rightarrow #4\rightarrow #5}
\renewcommand{\bar}{\overline}
\renewcommand{\tilde}{\widetilde}
\newcommand{\eps}{\epsilon}

\title[]{Counting Irreducible Representations of General Linear Groups and Unitary Groups}
\author{Qiutong Wang}
\institute{Zhejiang University}
\date{August 11, 2025}

\begin{document}

\begin{frame}
  \titlepage
\end{frame}

% Uncomment these lines for an automatically generated outline.
%\begin{frame}{Outline}
%  \tableofcontents
%\end{frame}



\begin{frame}{Outline}

\begin{itemize}
  \item Introduce the counting method developed by Dan Barbasch, Jia-Jun Ma, Binyong Sun, and Chen-Bo Zhu in their paper: Special unipotent representations of real classical groups: Counting and reduction. J. Eur. Math. Soc. (2025). 
  \item Show it's application in counting the irreducible representations of general linear groups and unitary groups.
\end{itemize}
\end{frame}












\begin{frame}{Background}
  \begin{itemize}
    \item $\RG$: connected reductive algebraic group defined over $\BR$;
    \item $G$ is a real Lie group together with a Lie group homorphism $\iota: G \to \RG(\BR)$ with open image and finite kernel;
    \item $\fg$, $\fg_0$ are Lie algebras of $\RG(\BC)$, $G$;
    \item $^{a}\fh$: the abstract Caratan subalgebra of $\fg$, with root lattice $Q_{\fg}$, weight group $Q^{\fg}$, and analalytic weight lattice $Q_{\iota}$ ($Q_{\fg} \subseteq Q_{\iota} \subseteq Q^{\fg} \subseteq {^{a}\fh}^*$);
    \item $W$: the abstract Weyl group of $\fg$ act on $^{a}\fh$;
    \item $\mathrm{Rep}(G)$: the category of Casselman-Wallach representations of $G$;
    \item $\mathrm{Irr}(G)$: set of isomorphism classes of irreducible objects in $\mathrm{Rep}(G)$.
  \end{itemize}
  




\end{frame}






\begin{frame}

  \begin{itemize}
    \item There is a partition of $\Irr(G)$ with respect to infinitesimal characters:
  \[\Irr(G) = \bigsqcup_{\lambda \in W \backslash {^{a}\fh}^*} \Irr_{\lambda}(G),
  \]
  according to work of Harish-Chandra, each set $\Irr_{\lambda}(G)$ is finite;
  \item According to complex associated variety (annihilator variety), there is a further partition of $\Irr_{\lambda}(G)$:
  \[\Irr_{\lambda}(G) = \bigsqcup_{\CO \in \RG(\BC)\backslash \Nil(\fg)}\Irr_{\lambda}(G;\CO).\]
  \end{itemize}

Goal: Describe the size of each set $\Irr_{\lambda}(G;\CO)$ in terms of combinatorial data.
\end{frame}






\begin{frame}{Counting Formula}
\begin{block}{Theorem (Barbasch, Ma, Sun, Zhu)}
  \begin{equation*}
        \sharp(\Irr_{\nu}(G;\CO)) \leq \sum_{\sigma \in \Irr(W(\Lambda);\CO)} [1_{W_\nu}:\sigma] \cdot [\sigma:\mathrm{Coh}_{\Lambda}(\CK(G))],
    \end{equation*}
    where $1_{W_\nu}$ denotes the trivial representation of the stabilizer $W_\nu$ of $\nu$ in $W$. The equality holds if the Coxeter group $W(\Lambda)$ has no simple factor of type $F_4$, $E_6$, $E_7$, or $E_8$, and $G$ is linear or isomorphic to a real metaplectic group.
\end{block}

In their paper, they use this formula to count the number of special unipotent representations in order to construct them via theta correspondence.

\end{frame}





\begin{frame}{Double Cells, Special Representations}
  \begin{itemize}
    \item 
  \end{itemize}
\end{frame}




\begin{frame}{Coherent Continuation Representation}
  $\CR_{\mathrm{hol}}(\RG(\BC))$: Grothendieck ring of finite-dimensional holomorphic representations of $\RG(\BC)$.\par
  $\CK(G)$: Grothendieck group of $\Rep(G)$ which has a $\CR_{\mathrm{hol}}(\RG(\BC))$ module structure via tensor product.
  \begin{block}{Coherent family}
    Let $\Lambda = \nu + Q_{\iota} \subseteq {^{a}\fh}^*$, a $\Lambda$-coherent family is a map
   $$\Phi: \Lambda \to \CK(G),$$
   such that:
   \begin{itemize}
      \item for any $\nu \in \Lambda$, $\Phi(\nu) \in \CK_{\nu}(G)$,
      \item for any $F \in \CR_{\mathrm{hol}}(\RG(\BC))$ and $\nu \in \Lambda$, $F \cdot (\Phi(\nu)) = \sum_{\mu \in \Delta(F)} \Phi(\nu + \mu)$ (where $\Delta(F)$ is the set of weights of $F$ counted multiplicity).
   \end{itemize}
  \end{block}
\end{frame}






\begin{frame}
  Let $W(\Lambda) \subseteq W$ denote the integral Weyl group with respect to $\Lambda$.
  \begin{block}{Coherent continuation representation}
  Let $\mathrm{Coh}_{\Lambda}(\CK(G))$ denote the complex vector space of all coherent families on $\Lambda$. It is a representation of $W(\Lambda)$ under the action
  \[(w \cdot \Psi)(\nu) = \Psi(w^{-1}\nu),\]
  for any $w \in W(\Lambda)$, $\Psi \in \mathrm{Coh}_{\Lambda}(\CK(G))$, $\nu \in \Lambda$.
  \end{block}
    For any $\nu \in \Lambda$ we have the evaluation map \defmap{\mathrm{ev}}{\mathrm{Coh}_{\Lambda}(\CK(G))}{\CK_{\nu}}{\Psi}{\Psi(\nu)}
  \begin{block}{Theorem (Schmid, Zuckerman)}
    $\mathrm{ev}$ is surjective for each $\nu \in \Lambda$, and bijective when $\nu$ is regular.
  \end{block}

\end{frame}






\begin{frame}{Harish-Chandra Cells}
  \begin{block}{Theorem(Vogan's green book)}
    Suppose $\nu \in {^{a}\fh}^*$ dominant, $M \in \CK_{\nu}(G)$ is an irreducible representation representation. Then there exist a unique coherent family $\bar{\Psi}$ characterised by the following properties:
    \begin{itemize}
      \item $\bar{\Psi}(\nu) = M$;
      \item If $\mu$ is dominant, then $\bar{\Psi}(\mu)$ is irreducible or zero.
    \end{itemize}
  \end{block}
  There is a basis $\CB = \{\bar{\Psi_i}\}$ of $\mathrm{Coh}_{\Lambda}(\CK(G))$ such that for any regular dominant $\mu$, $\bar{\Psi_i}(\mu)$ is an irreducible representation with infinitesimal character $\nu$.\par
  We call $(\mathrm{Coh}_{\Lambda}(\CK(G)),\CB)$ a \textcolor{red}{basal representation} .\par
  We can also define basal subrepresentations. 
\end{frame}






\begin{frame}
  For any subset $\CS$ of $\mathrm{Coh}_{\Lambda}(\CK(G))$, denote by $\left \langle \CS \right \rangle$ the minimal basal subrepresentation containing $\CS$.\par
  Define an equivalence relation on $\CB$ by: 
  $\bar{\Psi_i} \approx  \bar{\Psi_j}$ if and only if $\left \langle \bar{\Psi_i} \right \rangle = \left \langle \bar{\Psi_j} \right \rangle$.\par
  The equivalence classes of $\CB$ with respect to this relation are called Harish-Chandra cells.
  \begin{block}{Cell representations}
    Let $\CC$ be a cell in $\CB$ and put $\bar{\CC} = \left \langle \CC \right \rangle \cap \CB$. Define the cell representation attached to $\CC$ by
    \[ \mathrm{Coh}_{\Lambda}(\CK(G))(\CC) := \left \langle \bar{\CC} \right \rangle  / \left \langle \bar{\CC} \backslash \CC \right \rangle\] 
  \end{block}

  
\end{frame}





\begin{frame}
  \begin{block}{Hypothesis}
    The set \set{$\sigma \in \Irr(W(\Lambda))$}{ \textrm{$\sigma$ occurs in $\mathrm{Coh}_{\Lambda}(\CK(G))(\CC)$}} is contained in the double cell containing the special representation $\sigma_{\CC}$.
  \end{block}
\end{frame}




\begin{frame}{BMSZ's Proof}
  
  \begin{align*}
    \sharp(\Irr_{\nu}(G)) &= \dim \CK_{\nu}(G) \xlongequal{Vogan} \dim \mathrm{Coh}_{\Lambda}(\CK(G))_{W_\nu} \\
    & = [1_{W_{\nu}}:\mathrm{Coh}_{\Lambda}(\CK(G))] \\
    & = \sum_{\sigma \in \Irr(W(\Lambda))} [1_{W_\nu}:\sigma] \cdot [\sigma:\mathrm{Coh}_{\Lambda}(\CK(G))],
  \end{align*}
  If $S$ is a Zariski closed $\RG(\BC)$-stable subset of $\Nil(\fg)$, then
  \begin{align*}
    \sharp(\Irr_{\nu,S}(G)) & = \sum_{\sigma \in \Irr(W(\Lambda))} [1_{W_\nu}:\sigma] \cdot [\sigma:\mathrm{Coh}_{\Lambda,S}(\CK(G))]\\
    & \xlongequal{\textcolor{red}{Hypothesis}} \sum_{\sigma \in \Irr_{S}(W(\Lambda))} [1_{W_\nu}:\sigma] \cdot [\sigma:\mathrm{Coh}_{\Lambda}(\CK(G))].
  \end{align*}
\end{frame}











\begin{frame}{Combinatorial Notations}

\end{frame}


\end{document}
